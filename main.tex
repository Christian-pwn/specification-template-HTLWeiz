% Template for specification of a thesis (HTL Weiz edition)
% by Christian Schorna, 2022
\documentclass[12pt]{article}
\usepackage{specStyle}

\pagestyle{fancy}
\cfoot{\thepage\ | \pageref{LastPage}}
\renewcommand{\headrulewidth}{0mm}


\begin{document}

%
%
%
\begin{titlepage}
    \samepage
    {
        \begin{center}
            \includegraphics[width=40mm]{Figures/HTLlogo}

            \vspace{5mm}
            \textbf
            {
                Höhere Technische Bundeslehranstalt Weiz \\
                Abteilung für Elektrotechnik
            }

            \vspace{10mm}
            \huge{\textbf{Pflichtenheft}}
        \end{center}

        \vspace{15mm}
        \renewcommand{\arraystretch}{1.5}
        \begin{tabular}{l @{\hspace{40mm}}l}
            \textbf{Thema:} & \textbf{Lastenlift} \\
            \textbf{Jahrgang:} & 4BHET \\
            \textbf{Schuljahr:} & 2022 \\
        \end{tabular}
    }
\end{titlepage}

\section*{Projektteam}
\begin{center}
    \begin{tabular}{| l | l | l | l |}
        \hline
        \textbf{Name} & \textbf{Individuelle Themenstellung} & \textbf{Telefon} & \textbf{E-Mail} \\
        \hline
        Nachname Vorname &  & +43 xxxxxxxxxx & E-Mail \\
        \hline
        Nachname Vorname &  & +43 xxxxxxxxxx & E-Mail \\
        \hline
    \end{tabular}
\end{center}

\section*{Betreuer/innen}
\begin{center}
    \begin{tabular}{| l | l | l | l |}
        \hline
        \textbf{Rolle} & \textbf{Name} & \textbf{Telefon} & \textbf{E-Mail} \\
        \hline
        Hauptverantwortlich & Nachname Vorname (KZ1) & +43 xxxxxxxxxx & E-Mail Adresse \\
        \hline
         & Nachname Vorname (KZ2) & +43 xxxxxxxxxx & E-Mail Adresse \\
        \hline
    \end{tabular}
\end{center}

\section*{Projektpartner}
\begin{center}
    \begin{tabular}{| l | l | l | l |}
        \hline
        \textbf{Firmenname} & \textbf{Name} & \textbf{Telefon} & \textbf{E-Mail} \\
        \hline
         & Nachname Vorname & +43 xxxxxxxxxx & E-Mail Adresse \\
        \hline
    \end{tabular}
\end{center}

\section*{Versionskontrolle}
\begin{center}
    \begin{tabular}{| l | l | l | l |}
        \hline
        \textbf{Version} & \textbf{Datum} & \textbf{Autor(en)} & \textbf{Änderungsgrund / Bemerkungen} \\
        \hline
        0.1 & 01.01.2022 & KZ1, KZ2 & Ersterstellung als Diskussionsgrundlage \\
        \hline
        0.2 & 01.01.2022 & KZ1 & Erste Revision \\
        \hline
        1.0 & 01.01.2022 & KZ2 & Endversion \\
        \hline
        1.1 & 01.01.2022 & KZ1, KZ2 & Endversion als Diskussionsgrundlage \\
        \hline
    \end{tabular}
\end{center}
\clearpage

\fancyhead{}
\tableofcontents
\clearpage

\section{Allgemeines - Zweck und Ziel dieses Dokuments}
In diesem Dokument wird festgehalten, welche Arbeiten im Rahmen des Projekts zu erledigen sind.
Dazu werden konkrete Fälle explizit ein- oder ausgeschlossen.
Zur besseren Übersicht befindet sich im Anhang eine stichwortartige Aufzählung der zu erledigenden Arbeiten.
Die Anforderungen sind dabei in Muss- und Optionale Punkte unterteilt. \\
\textbf{Auf die Implementierung wird im Pflichtenheft nicht eingegangen.}

\section{Informationen für die Projektdatenbank}
\subsection{Ausgangslage}
Beschreibung der IST Situation, der Problemstellung bzw. den Anlass für das Projekt.
Die Beschreibung der Ausgangslage soll verständlich machen warum das Vorhaben relevant und notwendig ist (z.B. ist das Projekt innovativ, gibt es bereits ähnliche Projekte, kann auf bereits gesammelte Erfahrungen aufbaut werden, usw.).

\subsection{Partner und Betreuungspersonen}
Beschreibung der Partnerinstitutionen und deren Vertreter.

\subsection{Untersuchungsanliegen der individuellen Themenstellung}
Hier wird erörtert, was durch dieses Projekt z.B. beschrieben, gebaut, überprüft oder geklärt werden soll.
Es soll auch beschrieben werden, wie das Projekt durchgeführt wird und welche konkreten Aktivitäten geplant sind. \\
Individuell pro Themenstellung!!

\subsection{Zielsetzung}
Beschreibung der beabsichtigten Ziele des Projektes.
Die Zielsetzung gibt Auskunft darüber, was Sie mit dem Projekt erreichen möchten.
Die Norm definiert das Projektziel als „Gesamtheit von Einzelzielen, die durch das Projekt erreicht werden“.

\subsection{Geplantes Ergebnis der Prüfungskandidaten/des Prüfungskandidaten}
Bei der Definition der geplanten Ergebnisse ist es sinnvoll, sich auch über deren Überprüfbarkeit Gedanken zu machen, also Indikatoren zu überlegen, an Hand derer man die Erreichung des Zieles erkennen kann.
Es sollen konkrete, messbare Maßnahmen definiert werden. \\
Individuell pro Themenstellung!!

\subsection{Projektbezug}
Hier wird die Grundidee des Projekts beschrieben;
in welchem Umfeld Sie eingesetzt wird, wozu sie gut ist, etc.

\section{Zielkriterien}







\end{document}